\chapter{To Dad}

\noindent Hey Dad,

\sout{How are you today?}\footnote{Ok, you can do better}

\sout{I hope all is well!}\footnote{Too peppy...}

\sout{I've been going to therapy. I was told to write down everything I thought of related to the...}\footnote{He doesn't need to know \textit{why} I am writing this...}

Right, ok.

I obviously have no idea how to write this, so let me try again.

\newpage

\section*{the secret}

\noindent Hey Dad,

I don't know where you are right now or what you have been doing with yourself for the past few years, but I've been well.
I mean, sure, Joanne is now out of the picture, but I like to think that I've rebounded since then and tried my best to raise my daughter the same way you raised me: true to ourselves, no matter who that might be.
I know you never met her, but I am sure you would love her almost as much as I do.

She's kind, compassionate, and wickedly smart.
She must get her intelligence from you or \sout{Joanne} you, because the Lord knows my IQ is about the same as... \footnote{make a clever joke.} 

It's been so long, there's so much I want to catch you up on, but maybe I should start (as I always do) with a long, rambly story with no discernable purpose.
This one happened on a cold November evening, freshman year of high school.

At that time of year, the occasional snowstorm was not uncommon, but blizzards were rare.
Well, as our luck would have it, we had a blizzard.
Snow was piling up outside, quickly blocking the doors and windows.
In a few hours, the entire town would be encased in ice.
I (of course) had locked myself away in some unknown location and was completely oblivious to all of it.

You, the rational adult, were frantically going door to door with a snow shovel in hand, asking all of my friends if they had seen me.
Of course no one had because I had purposefully hid myself away from the world in our storm cellar.
See, at the time, I had a bit of a secret hobby I didn't want anyone to know about.
As a teenager, you can only imagine what it was\footnote{Ok, c'mon. That sounds suggestive, but there was nothing sexual about it.}

Well, after about 2-3 hours of my secret activity by candle light, I found myself trapped.
There was so much snow on top of the door, I couldn't force it open.
Worse, the power was out so I couldn't turn on the lights.
At the time, I was worried that the snow would also have sealed any of the vents to the cellar, leaving me without oxygen, so I extinguished my candle and just sat there in complete and total darkness.
Now, I don't know if you remember, but our cellar was quite deep underground.
It was also decently large with a couch, table and chair set, and a cabinet of tools.
It was \textit{the} place my friends and I would go to hang out in complete and total privacy, which was great for a boy growing up, but absolutely awful in the situation I had found myself in.
I knew no matter how much I shouted, no one would hear me.

I would like to say I was brave in that moment, but I was not.
I panicked and began to cry in the fetal position in the middle of the room.
My entire life flashed before my eyes.
I then began to worry about what my friends and family would think when they found me weeks later, dead in the storm cellar.
To be honest, I don't know if I fully understand what I did next, but I am sure in that moment, it made perfect sense.
I began to clean the room.

I can only imagine I thought to myself, ``Well, if people are going to find me dead, at least they won't think I'm a slob when they do.''
The problem was that I was still terrified of running out of oxygen, so rather than lighting the candle again, I turned on the lighter for long enough to find my bearings, then pulled out the cleaning solution and got to work... Still in total darkness.
In my mind, I mopped the floors, scrubbed the walls, organized the tools.
I did everything I could think of.

In fact, I had gotten so into it, I had failed to hear a loud scraping against the cellar door.
In a few moments, the door opened, causing me to shield my eyes to the light.
I then saw you standing there, huffing and puffing with a shovel in hand, smelling a lot like dead fish.

I couldn't quite see the expression on your face, but I imagine it was right in the center of blind rage and thorough relief.
Then, I heard you chuckle a bit as the shovel fell to the ground.
I was crouched down with a sponge, scrubbing the floor in the blistering cold, no doubt about to suffer from hypothermia.
As I looked around, I realized my mistake.
The entire cellar was now an abstract orange painting.
Apparently in my panic, I had mixed up the cleaning solution with orange paint.
To this day, I still I don't really even know how I did that.

Worse, I had failed to clean up my secret activity.
Sprawled across the sofa were loose pieces of paper, all covered in poems.
I then panicked for a completely different reason and tried to collect them all back into my notebook, but that just meant my writing was now sticky and orange.

You then did something that changed me.
Rather than getting frustrated at the newly painted cellar or disappointedly calling me out for my somewhat feminine hobbies, you sighed and said, ``C'mon home. I've got some hot soup. We'll deal with the rest in the morning.''

We made sure to shut the cellar door behind us as tightly as we could and then basically waded through the snow back home.
The wind was howling, but my thoughts were clear.
I found myself embarrassed that I almost lost my life over a silly secret hobby.
The house was quiet, illuminated by small electric lanturns.
I am not sure if you said anything until we were properly seated at the dining table with some kind of fish stock in front of us.
You then looked at me and said, ``Well, thanks for painting the cellar. It was getting a little worse for wear. We can finish it up later."

We both laughed a bit and I could see you trying to piece together what you just saw, but then your eyes darted to the couch where I had thrown my notebook.
``So what were you up to in the cellar in the first place? Writing your will? I would hope I'm in there after all the stress you put me through this afternoon!''
He chuckled, obviously intending it as a joke, but it landed a bit flat.

After a brief, awkward pause, I couldn't find the right words to answer.
See, I always saw you as the epitome of manliness and was ashamed that I couldn't live up to the same image.
In my entire life, I had never seen you even open a book and you always said, ``reading is for those who have too much time on their hands.''
I mean, what would you think if your only son decided he wanted to run off to become a poet?

You then continued a bit, ``Look, Lars. You obviously weren't doing anything wrong.
The orange paint's a bit weird, but I assume you didn't do that on purpose.
I also respect your privacy.
Feel free to use the cellar for whatever you want, whenever you want, but I'm a bit sad that I created an environment where you feel the need to keep secrets.''

I then felt ashamed for an entirely different reason and told you the real reason I was there that afternoon, ``Poetry.''

Somehow, in my brain, I found myself bracing for impact -- as if I just revealed some critical flaw in my character, but you instead engaged with it and said,
``Poetry? That's actually pretty cool. I always took you for more of a fantasy or sci-fi author, myself...''

We both laughed a little before I continued, ``Yeah, Poetry. I found Walt Whitman's \textit{Leaves of Grass} in the library and thought, `Well, I can do that.' and then just started writing.`` I paused a bit, thinking about what to say, ``As it turns out, it's a bit harder than it looks.''

You then said something rather insightful, ``Well, I imagine it is. I don't read too much poetry, but even I've heard of Whitman. He probably wrote all day and all night for years before he made something that stuck.''

I paused for a second, trying to piece together what had just happened.
I thought out loud, ``I guess I am just embarrassed because no one writes anymore.
None of my friends read anything outside of school, so I feel like I'm doing something wrong.''

``Well, you aren't.'' You said, ``In fact, I think you are doing something right.
When I was your age, I was really into weightlifting, but none of my friends were into that kind of thing.
If I had stuck to it, I could have gotten a 6 pack, but I didn't and now look at me!''
You then slapped your belly and chuckled to yourself.
``My point is that you are young. Now is the time to pick up productive hobbies, and when you feel up to it, feel free to share them with me!''

I sighed and didn't know what to say.
I mean, I \textit{should} have said, ``thanks for making me feel better,'' but my angsty teenage brain wouldn't allow me to do that, so I instead sat in silence.

You then looked to the notebook again and said, ``So, can we salvage it? or is it all orange now?''

``It's salvagable, I think...'', I said while picking it up and pulling out the loose pages, ``We just have to let it dry.''

You then helped me by clearing some counter space and laying down some newspaper.
Along the way, you then asked a simple question, ``So... None of your friends are readers?''

``No, not really.'' I replied.

``Then what's wrong with using the library from now on? It will be just as private as the cellar and slightly less orange.`` At this stage, you were trying to wash paint from your fingers. ``Also, take a shower as soon as the electricity comes back on. Otherwise, you might have orange highlights in your hair for the rest of the month.''

There was more to the story (there always is).
A few days later, I formally painted the cellar orange, but we kept the reason why as a secret between us.

Come to think of it, I don't know what lesson I was supposed to learn from this.

Was it that keeping secrets is bad?
If so, shouldn't we have been more open about the orange cellar?

Was it to stay out stop using the cellar entirely?
Well, I definitely used that space a bunch with friends.

The fact is that there are many more interactions just like this one that I could have pointed out, but it was in this moment that I realized we were always on the same team.
Even if you didn't fully understand my hobbies or interests, you encouraged me to keep going and making myself better and better.

To this day, one of my biggest regrets is never publishing a book.
It's been on my bucket list for ages, but life got in the way.
It always gets in the way.

Look, Dad.
I have so much to tell you, it's insane.
I want to tell you about Joanne -- the biggest mistake I have ever made in my life.
I want to tell you about Leia, who stands to-date as my biggest achievement.
I want to tell you everything.

But I've decided that maybe it's best to leave you to it for now and instead send them letters directly.

I guess my final word is:

\huge{Thanks}

\normalsize{-- Lars}
