\chapter{To Joanne}

\section*{scribblings}
\begin{enumerate}
\item Joanne is actually toxic towards Lars, causing him to have chronically low self-esteem, but he doesn't realize this until writing these letters
\item Joanne leaves after a huge argument and steals Leia, only to return her a few days later without any further contact.
\item She \textit{does} realize she is toxic but does not know how to fix herself
\end{enumerate}

Hey Joanne,

\sout{Leia is doing well.}

\sout{Can we talk?}\footnote{We both know I won't be sending this letter. I'm only writing it because the therapist recommended me to do it...}\footnote{and the only reason she recommended it was because it seemed to help me with my dad...}\footnote{so dad, it's your fault I'm writing this!}

Ok, look. Joanne.
I don't know what to say.

Leia's doing well.
She's just started high school, which means it's been almost a decade since you last saw her.
Man, I wish you were around during her angsty teenage phase, it was great!
Well, great in hindsight.

Honestly, it was a bit of a chore to go through, in-person, so you did the right thing to avoid it overall.

\begin{noteblock}{I'm treating her like a stranger. Maybe because she \textit{is} one?}
But that's the joy of parenting!
\sout{A joy you never wanted to have...}
\end{noteblock}

Hey.

Do you remember Rebecca? Rebecca Jorge? from high school?
You thought it was funny that their last name was another first name -- and a man's name at that.
You would often joke that they were trans, and I would go along with it.
You would always call them Jorge instead of Rebecca, and ask her if she was going in to the right restroom during breaks between classes.
I didn't say anything directly, but I kinda laughed a bit at the jokes, encouraging you to keep going.

Those moments haunt me to this day.

I sometimes can't sleep at night thinking about the way \sout{you} \textit{we} treated them.
I mean, I know we were kids -- what? 13 or 14?
We definitely stopped by the time we were sophomores.

Heck, I even went out of my way to apologize to them.
Afterwards, we realized we had a lot of things in common.
We both loved swimming and climbing, and she even read some of the same books.
Throughout the next year, I would hang out with them periodically until you explicitly told me to stop one afternoon.
We were about to walk home, and I asked if she wanted to join us.
You then grabbed my hand and said, ``She's not allowed to walk with us.''

I actually stood there, dumbfounded and asked what you meant. You then pushed me aside and spoke to them directly, ``Look, Jorge. Lars is only hanging out with you because he feels bad for you, not because he actually wants to be friends.''

You then grabbed my hand and tried to force me to walk with you.
I froze.
They were crying.
You then said, ``Look. It's either her or me. Let's go.''

So I went with you.

When we got home, you began to freak out, saying I shouldn't hang out with ``people like her.'' And that she was bringing be ``down to her level by forcing people to feel sorry for her.''

Whatever you said, it worked.
I no longer hung out with them.
To be honest, they never reached out to hang out with me either, which kinda hurt, but I pushed those feelings aside.
I had you as a friend, and that was good enough for me.

But we were teenagers then.
I'm 40 now.
You're 40.
We are old.

It's been a while, and I decided to reach out to them again.
As it turns out, they \textit{were} actually trans.
They just recently changed their name to Ryan.
It doesn't fix the double-first name issue, but we now regularly work out almost every morning before the kids go off to school.

To be honest, I don't know why I brought this story up.

I mean, you don't care.
Ryan doesn't care.
I don't know why I care anymore.

It's just that...
Well, I was talking to another friend recently.
One you might not have met yet.

I told him about our \textit{situation} and he said that some relationships are destined to fail for one reason or another, which is universally true...
But then he said something rather interesting, ``The best way to get over someone is to imagine them at their worst, and then imagine your life with the worst version of them.''

Now, my therapist actively condemned this idea, saying it might lead me towards depression given my other tendencies, but I still can't help reflecting on \textit{us} and wondering if there was some solution -- some way to play my cards just right so we were still together.

I guess hanging out with Ryan caused me to reflect specifically on our history together.

I mean, I realize that people change.
We have all changed.
I am a radically different person than who I was 6 months ago.
25 years ago, I was just some dumb teenager trying to make sense of the world.
I shouldn't treasure ancient memories more than the present.

But I just can't help it.
The teenage years are formative.
They are part of my history -- part of what made me who I am today.
\textit{You} are a big part of that.

Even today, I have certain mannerisms that remind me of you.
I'll incessantly apologize for everything, even when it's hardly at all my fault.
I'll triple-lock the door even though there hasn't been a break-in in this town for the better part of a century.
I'll pick apart people's clothing and facial expressions to find the \textit{perfect way} to get what I want out of them...

I guess I just find myself wondering if Leia would be better with her mom or not.
Right now, I don't have an answer.

