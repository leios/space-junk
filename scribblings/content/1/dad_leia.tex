It just strikes me as insane how long it takes for humans to mature.
I mean, when a dog is 3 months old, it's essentially able to do anything an adult dog can do.
When Leia was 3 months old, she didn't even know she had hands.

\begin{enumerate}
\item GreyBox
\item Valedictorian Speach
\item Rambling
\end{enumerate}

\chapter{The Graduation}

\noindent Hey Dad,

On the first day of school after summer break, Joanne and I had the tradition of meeting up half an hour before the opening bell to analyze our schedules and figure out the best meeting spots throughout the day.
We rarely had similar classes, but could sometimes find a few hallways to meet in throughtout the day.
Sometimes we even shared a lunch period.
One year, we were excited to find that we shared the same class, but neither of us knew what it was.
It was new and simply called ``Study Hall.''
Apparently, the idea was actually quite common a few centuries ago, but fell out of fashion for one reason or another.

We didn't know what to expect, but it sounded a lot like a free period, which was even more exciting.
After my intro to AI class, I bolted over to her quantum lecture and waited in the hallway while her class filtered out.
When I saw her, I tapped her shoulder and the two of us began chatting a bit while on our way to the new class.
We were joking about all the games or movies we would watch over the year during the break period, but when we entered the classroom, we saw something quite peculiar: the back row of desks were covered by giant, translucent boxes and closed off from the rest of the class.

The enclosures were apparently called GreyBoxes, and were new technology being pushed by some federal educational committees.
They were soundproof boxes that went around a few desks, allowing students to see both the classroom and tailored visuals via Augmented Reality (AR).
While the GreyBoxes were being installed, the school also introduced GreyRooms, which were AR classrooms for the teachers to use so they could stream to the GreyBoxes.

The idea was simple: students would be given an hour a week in the GreyBox, while teachers would be required to teach in the GreyRoom.
This put a layer between the teacher and student so they would no longer be in the same room -- which admittedly sounds like an unnecessary step towards learning.
The key advantage was that now students could learn from any teacher in any participating school in the world.

For us, the GreyBoxes were all installed over winter break, so when we returned to school in January, there was a sort of tepid excitement among the students who were simultaneously eager to try out the new technology and dreading yet another weird teaching gimmick to figure out.
The teachers were also whispering about how they were not \textit{quite} comfortable with teaching a random assortment of students for one of their classes every day.
The general consensus was that no matter how good the technology was, it could never replace in-person teaching...

But then I tried it, and I was \textit{immediately} sold.

Joanne and I had study hall after lunch, and we had heard mixed reviews about the GreyBoxes from fellow students.
Apparently, the boxes were really cool, but we could choose the classes being taught, so there was no real incentive to take a difficult class as these could negatively influence our GPA.
As we made our way to class we were joking to each other about how we would use the boxes to watch movies instead of actually learning anything.
When I entered the classroom at the start of the winter term, I saw them: weird glass structures encasing the back row of desks.
I was then told to check my phone to find the desk rotations, which would determine what day of the week I would be in the GreyBox row.
As luck would have it, On the very first day, I was set to sit in desk 13, which was a GreyBox desk.

So I awkwardly waved to Joanne and walked to the desk, opened the door, and sat down.
I then awkwardly waved to Joanne again from behind the grey glass.
I plugged my phone in to the pocket on the side of the desk and the room immediately lit up.
I could still kinda see the rest of the classroom, but I had to really squint to see any details.

I saw a small graphic appear directly in front of me and an AI voice began to speak, ``Larson Lochston. Sophomore. Is this correct?"

I looked around a bit, again squinting to see if I could see anyone.
I couldn't, but seeing as I \textit{was} still in study hall and supposed to be a bit quiet, I whispered back, ``yes.''

The AI responded, ``Ok. Please feel at ease to speak as loud as you would like. The GreyBox is completely soundproof. No one can hear anything you say; however, you will be monitored. If you say anything concerning, such as oppressive or cruel remarks about classmates or exhibit frustration at the learning material, we reserve the right to send this information to your student counselor. Is this ok?''

I cleared my throat and said, ``yes.''

The AI continued, ``Great. Remember, that the GreyBox is a tool meant to enhance your learning experience. You will be paired with distributed teachers who best suit you, as an individual. If you ever feel we have paired you incorrectly or otherwise feel that your learning is inadequate, please let us know before class so we can re-assign you to a new teacher. Are there any questions so far?''

``I guess not?'' I responded.

``Great. Now to begin the pairing process. Your school has opted in to the elective program, this means that the GreyBox will prioritize classes for you that are outside of the curriculum traditionally offered by your school. Keep in mind that if you wish, the GreyBox can also be used as a form of remediation to help you with specific classes. In your case this is unnecessary as your grades are sufficiently high. If your grades drop below a predetermined threshold, the GreyBox will enter remediation mode and you will switch to the remediation track. Does this make sense?''

``Yes.''

``Ok. The school defaults the remedial path to begin if any of your primary classes drop below a C average. Would you like to change this default?''

I paused for a second, realizing that this might have been an oversight by the school.
I could have very easily told the AI to switch the default to ``F'', and then pick the easiest possible class.
This would guarantee an easy study hall for the term -- or at least until the bug was patched out.

But then I realized something else.
Whoever I get as an instructure \textit{could} be better for me than the instructor I have in-person.
If that was the case, I would like for the remedial mode to bump in as soon as possible to make sure I didn't fall behind on classes that mattered more.
So I told the AI, ``Please change the default to A.''

``So you would like us to provide remedial classes to you if your grade in any class drops below an `A'. Is this correct?''

``Yes.''

``Great! By the looks of it, none of your classes meet the criteria for remedial classes, so we will now seek to find an elective of your choosing. Do you have a preference for something you would like to learn?''

My mind then immediately raced to all the afternoons in the library writing and (of course) the infamous orange cellar incident.
Again, though, I had not shared my hobby with anyone and was still quite embarrassed about it, so instead of answering the question directly, I quietly asked, ``So, no one can hear me, right?''

``That is correct.''

``Can I learn... Uh... Poetry?'' I remember being really embarrassed at that moment and stared at my hands while twiddling my thumbs.

``Yes. Poetry is available, certainly. It seems you prefer exercise-based learning. Is this correct?''

``Yes. I would prefer if I could actually write something.''

``Ok, I would suggest a class in creative writing. Is this alright?''

``Yes, that sounds perfect.''

``Ok. There is a class being taught by Dr. Ritlin in from Monteneaux City Montanna. Is this ok?''

``I guess?''

``Please remember you can change intructors whenever you want if you so desire. I will now route you to Dr. Ritlin's class.''

The box flickered a bit and the room around me began to change.
A potted plant appeared in the window.
The row of desks in front of me became clear, and I was surprised to find a row of desks appear behind me as well.
The walls became plastered with a bunch of large letters with quotes from famous historical novels.

Soon, other students began filtering in.
First a girl appeared to my left.
By her shocked expression, I could tell we were both equally surprised to see each other.
I glanced around, wondering if she could hear me speak.
I then whispered, ``Hello?''

To which she responded, ``Wait, we can talk to each other?''

A student a few rows in front of us laughed a bit, ``I said the same thing the first time I used one of these. I'm Karl.''

The girl then introduced herself as Mary and I introduced myself as well.

Introduce Dr. Ritlin.

I actually loved the class so much that I deeply regretted my decision to keep my GPA at an A level.
If any of my grades dropped in any class, I would no longer be able to take Dr. Ritlin's class.

Honestly, by the end of my tenure in high school, I was sold on the technology.
Remote lessons that are tailored for each student seemed perfect.

Nowadays, every single lecture is taught via GreyBoxes.

It makes it all the more impressive that Leia not only passed all her classes with flying colors, but she was also valedictorian of her class.
In fact, she was nationally ranked as one of the top 100 students nation-wide!

It's been a few years since my last letter.

I don't really know what to say except that yesterday, Leia graduated from high school.
She's in the workshop, humming while collecting her things.
She will be getting on the pod to Boston next week.

\sout{I'm a wreck. I mean, not a \textit{wreck} wreck, but...}

\sout{I've got a moment of silence and thought I would write you another letter as a way to collect my own thoughts...}

\sout{I tried to convince her to stay and take classes remotely...}

I have actually been sitting here for hours trying to figure out what to write.
On the one hand, I realize that finishing high school is just one of many steps Leia will take in her own personal development.
It's not even a big step.
I mean, she finished all her high school courses ages ago and has been in an accelerated learning curriculum for almost three years.
She tells me she's actually got enough college credits to finish a physics degree at almost any US university.

By all accounts, she is doing great!
She was even valedictorian for her class!

I know that no matter what she chooses to do, she will succeed, it's just that...
Well, I was there in the hospital room when she was born.
The nurses, doctor, and I were all huddled around, anxiously awaiting Leia's arrival.
Joanne was obviously in pain, but I held on to her hand tightly and synchronized my breathing with hers as a way to subconsciously tell her we were in it together.

At the time, I questioned Joanne's decision to go through with a traditional birth instead of using a birthing pod, but I think I understand it now.
Maternity is a marathon -- a test of your metal as a human, and there's a sense of pride in accomplishing it.
No matter how far technology advances, humans are still human.
There are things we just want to experience as a way to prove our own worth.

Leia's delivery went without any major complications and immediately afterwards, the nurse asked me to cut her cord.
I was a bit too nervous to do it, fearing I would accidentally cut Leia, herself, with the scissors.
The nurse clearly saw my hesitation and wanted to motivate me to actually do it, so she told me it would be ``easy, just like cutting squid.''
To be honest, I am still a bit perplexed by that statement, but I \textit{did} end up cutting the cord.
I'm sorry to say that I never ate squid again.

They say that dads take a while to recognize their own babies.
When women have children, there are so many signs that prove their child is theirs.
I mean, there's the 9 months of pregnancy and all the hormones that come with it.
There's the fact that their body is producing milk for the first time.
They are the most important character in their child's life for at least a brief moment in time.

Men don't get that.
To them, they watched their partner go through an uncomfortable stretch of time, followed by an incredibly painful experience, resulting in an alien they are expected to form a deep connection with overnight.

Don't get me wrong.
Leia's great.
She's always been great.
But there was a time she was a stranger in her own home -- in \sout{my} our home.
A time when the only two people she could rely on were me and Joanne, and I was struggling to form an emotional bond with her entirely.

They say that between 6 and 25% of women suffer from post-partum depression \footnote{ADD CITATION}.
Did you know that roughly 10% of men suffer from similar symptoms \footnote{ADD CITATION}?

I know, I know.
I just said that men don't have the strong physical and emotional attachment to their children in the same way women do, but after a few weeks, I got the hang of things.
Leia was my child just the same as she was Joanne's.

I found myself cheering for her when she did the smallest of things: lifting her head up on her stomach, finding her thumb to suck on, grabbing a toy for the first time.
Now, she's the best student in her class and going to her dream university.
For someone her age, she's accomplished so much, and yet I still remember back to those moments when couldn't even lift her head.

Ah, man.
I'm getting sappy.
Might as well tap me for syrup at this point.

I've shared a bunch of stories with you since then and will hopefully continue to share more in the future.
Somehow, today, I'm reflecting on her entire life -- every single thing we have ever done together.

I don't know if you remember, but during my high school, they introduced the GreyBox Augmented Reality (AR) systems to teach everyone.
It was essentially a virtual classroom inside of a classroom.
Each desk was surrounded by a translucent AR screen that allowed them to see the other students but also receive a customized teaching experience to make sure that the students could learn at their own pace.
The idea was meant to allow children to keep all the social aspects of school while optimizing the educational components.
To put it simply: each class had the same students in it, but every student had a different teacher.

During my time, the GreyBoxes were only used during study hall as a way to catch students up on subjects they needed to improve on.
When Leia was in school, almost every lesson was in the GreyBox environment, which meant she was learning from teachers all over the country every single day.
Apparently, she 

Sometimes, groups were created based on similar curricula, but Leia was so far ahead she never had a group...

This lead to some strange interactions...

I know the graduation is just a small step for her, but it's a huge deal to me.
It means she'll move on.
She's no longer dependent on me.
Honestly, I feel like I was more dependent on her than she was on me.
